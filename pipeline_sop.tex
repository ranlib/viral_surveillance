\documentclass[11pt]{article}

\usepackage[margin=1in]{geometry}
\usepackage{setspace}
\usepackage{hyperref}
\usepackage{longtable}
\usepackage{booktabs}
\usepackage{listings}
\usepackage{xcolor}
\usepackage{amsmath}

\definecolor{codegray}{gray}{0.95}

\lstset{
  backgroundcolor=\color{codegray},
  basicstyle=\ttfamily\small,
  frame=single,
  breaklines=true
}

\title{Standard Operating Procedure (SOP)\\
Illumina Viral Surveillance Panel Analysis Pipeline}
\author{Dieter Best, Ph.D.}
\date{\today}

\begin{document}
\maketitle
\doublespacing

\section{Purpose}

This document describes the Standard Operating Procedure (SOP) for running the
Illumina Viral Surveillance Panel analysis pipeline on Linux systems.
The pipeline performs:

\begin{itemize}
  \item Sequencing quality control
  \item Host read removal
  \item Viral detection
  \item Variant calling
  \item Phylogenetic-ready consensus generation
  \item Cohort-level quality control reporting
\end{itemize}

The pipeline is implemented in Workflow Description Language (WDL) and is fully containerized using Docker images.



\section{Scope}

This SOP applies to:

\begin{itemize}
  \item Illumina Viral Surveillance Panel libraries
  \item Paired-end Illumina sequencing data
  \item Linux systems with Docker and a WDL engine (Cromwell recommended)
\end{itemize}

The pipeline is suitable for clinical surveillance, public-health monitoring, and research use.

\section{Pipeline Overview}

The pipeline processes samples independently using a scatter–gather model.
Each sample undergoes the following stages:

\begin{enumerate}
  \item Raw read quality control (FastQC)
  \item Host alignment and contamination estimation
  \item Host read removal (depletion)
  \item Viral classification (Kraken2)
  \item Viral alignment (BWA)
  \item Variant calling
  \item Variant quality control (bcftools stats)
  \item Global cohort-level reporting (MultiQC)
\end{enumerate}

\section{Software Requirements}

\subsection{Operating System}

\begin{itemize}
  \item Linux (Ubuntu 20.04 or newer recommended)
\end{itemize}

\subsection{Required Software}

\begin{itemize}
  \item Docker
  \item Java 11+
  \item Cromwell (WDL execution engine)
\end{itemize}

Example installation (Ubuntu):

\begin{lstlisting}
sudo apt update
sudo apt install -y docker.io openjdk-11-jre
\end{lstlisting}

\section{Reference Databases}

The pipeline requires the following reference resources:

\begin{itemize}
  \item Host reference genome (FASTA + full BWA index)
  \item Viral reference genomes (FASTA)
  \item Kraken2 viral database
\end{itemize}

All BWA index files must be provided explicitly:

\begin{itemize}
  \item \texttt{.amb}
  \item \texttt{.ann}
  \item \texttt{.bwt}
  \item \texttt{.pac}
  \item \texttt{.sa}
\end{itemize}



\section{Pipeline Inputs}

Per sample:

\begin{itemize}
  \item Paired FASTQ files (\texttt{R1}, \texttt{R2})
  \item Sample identifier
\end{itemize}

Global inputs:

\begin{itemize}
  \item Host reference and index files
  \item Viral reference database
  \item Kraken2 database directory
\end{itemize}

Inputs are supplied via a JSON file.



\section{Running the Pipeline}

\subsection{Example Command}

\begin{lstlisting}
java -jar cromwell.jar run viral_pipeline.wdl \
  --inputs inputs.json
\end{lstlisting}



\section{Pipeline Stages (Detailed)}

\subsection{Raw Read Quality Control}

FastQC is run on all input FASTQ files to assess:

\begin{itemize}
  \item Base quality
  \item GC content
  \item Adapter contamination
\end{itemize}



\subsection{Host Alignment}

Reads are aligned to the host genome using BWA-MEM.
The resulting BAM is coordinate-sorted and indexed.

This BAM is used exclusively for QC and host contamination estimation.



\subsection{Host Contamination and Viral Fraction}

Host contamination is defined as:

\[
\text{Host contamination (\%)} = \frac{\text{host-mapped reads}}{\text{total reads}} \times 100
\]

Viral fraction is calculated as:

\[
\text{Viral fraction (\%)} = 100 - \text{host contamination (\%)}
\]

An automatic QC flag is assigned:

\begin{center}
\begin{tabular}{lll}
\toprule
Viral Fraction & QC Flag & Interpretation \\
\midrule
$\geq$ 10\% & PASS & Adequate viral signal \\
1--10\% & WARN & Low viral signal \\
$<$ 1\% & FAIL & Likely negative or failed enrichment \\
\bottomrule
\end{tabular}
\end{center}



\subsection{Host Read Removal}

Reads where \textbf{both mates are unmapped to host} are extracted using
\texttt{samtools} and converted back to paired FASTQ files.

Duplicate removal is intentionally \textbf{not performed}, as targeted viral
panels frequently rely on PCR duplication for sensitivity at low viral load.



\subsection{Viral Detection (Kraken2)}

Host-depleted FASTQs are classified using Kraken2 with a viral reference database.
Summary reports are generated for downstream QC aggregation.



\subsection{Viral Alignment}

Reads are aligned to viral reference genomes using BWA.
The pipeline separates:

\begin{itemize}
  \item Alignment
  \item SAM to BAM conversion
  \item Sorting (coordinate and name)
  \item Indexing
\end{itemize}

This modular structure improves reproducibility and debugging.



\subsection{Variant Calling}

Variants are called from the viral BAM using bcftools.
The resulting VCF is bgzipped and indexed.



\subsection{Variant Quality Control}

\texttt{bcftools stats} is run on each VCF to generate:

\begin{itemize}
  \item SNP and indel counts
  \item Ts/Tv ratios
  \item Depth and quality distributions
\end{itemize}

These metrics are aggregated automatically by MultiQC.



\subsection{Global MultiQC Report}

A single MultiQC report is generated across all samples, including:

\begin{itemize}
  \item FastQC summaries
  \item Host contamination and viral fraction
  \item Automatic QC flags
  \item Samtools alignment statistics
  \item Coverage metrics
  \item Kraken2 summaries
  \item bcftools variant statistics
\end{itemize}

This report is the primary QC deliverable.



\section{Pipeline Outputs}

Per sample:

\begin{itemize}
  \item Host-depleted FASTQ files
  \item Viral BAM and index
  \item Variant VCF and index
  \item Host contamination and QC metrics (TSV)
\end{itemize}

Cohort-level:

\begin{itemize}
  \item MultiQC HTML report
  \item MultiQC data directory
\end{itemize}



\section{Quality Control and Interpretation}

\begin{itemize}
  \item Samples flagged \texttt{FAIL} should not be used for downstream analysis
  \item \texttt{WARN} samples may require repeat sequencing or review
  \item Coverage and variant QC should be interpreted in the context of viral load
\end{itemize}



\section{Notes and Limitations}

\begin{itemize}
  \item Duplicate reads are not removed due to lack of UMIs
  \item Viral fraction is an approximation for panel-based data
  \item This pipeline is not designed for de novo viral discovery
\end{itemize}



\section{Versioning and Validation}

All software versions are container-pinned.
Pipeline changes must be documented and revalidated prior to production use.



\section{Contact}

For pipeline questions, maintenance, or validation documentation, contact the pipeline maintainer.

\end{document}

